% !TEX encoding = UTF-8 Unicode
% -*- coding: UTF-8; -*-
% vim: set fenc=utf-8
\documentclass[a4paper,12pt,french]{article}

\usepackage{centrale}

\hypersetup{
    pdftitle={Titre à insérer},
    pdfauthor={Auteur à insérer},
    pdfsubject={Sujet à insérer},
    pdfproducer={Conversion PDF à insérer},
    pdfkeywords={Quelques mots-clés à insérer} %
}

\DeclareGraphicsRule{.ai}{pdf}{.ai}{} % pour insérer des documents .ai
\graphicspath{ {./img/} {./eps/}} % pour ne pas avoir à ajouter eps/ton-image.jpg


% ------------- Packages de dernière minute à insérer ici ------------- 


\begin{document}

% --------------------------------------------------------------
%                       Page de garde
% --------------------------------------------------------------

\begin{titlepage}
\begin{center}

\includegraphics[width=1\textwidth]{logo_ECL_rectangle_quadri_vectoriel.ai}\\[1cm]

{\large Intitulé de la filière, domaine ou approfondissement}\\[0.5cm]

{\large Type de projet}\\[0.5cm]

% Title
\rule{\linewidth}{0.5mm} \\[0.4cm]
{ \huge \bfseries Titre du rapport éventuellement en plusieurs lignes \\[0.4cm] }
\rule{\linewidth}{0.5mm} \\[1.5cm]

% Author and supervisor
\noindent
\begin{minipage}{0.4\textwidth}
  \begin{flushleft} \large
    \emph{Auteurs :}\\
    M. Prénom \textsc{Nom}\\
    M. Prénom \textsc{Nom}\\
    M\up{me} Prénom \textsc{Nom}\\
    M. Prénom \textsc{Nom}
  \end{flushleft}
\end{minipage}%
\begin{minipage}{0.4\textwidth}
  \begin{flushright} \large
    \emph{Encadrants :} \\
    Pr.~Prénom \textsc{Nom}\\
    Dr.~Prénom \textsc{Nom}
  \end{flushright}
\end{minipage}

\vfill

% Bottom of the page
{\large Version 0.1 du\\ \today}

\end{center}
\end{titlepage}

% --------------------------------------------------------------
%                    Table des matières 
% --------------------------------------------------------------

\thispagestyle{empty}
\tableofcontents
%\newpage

% --------------------------------------------------------------
%                         Début du corps
% --------------------------------------------------------------

\section{Introduction}


\newpage
\section{Une partie}

\subsection{Une sous-partie}

\lipsum[1]

\begin{figure}[ht!]
    \centering
    \includegraphics[width=0.3\textwidth]{example-image-a}
    \caption{Insérer ici le sous-titre.}
    \label{fig:id-de-la-figure}

\end{figure}



% --------------------------------------------------------------
%                            Partie 2
% --------------------------------------------------------------

\section{Une autre partie}

\subsection{Une autre sous-partie}
\lipsum[2]

% --------------------------------------------------------------
%                            Conclusion
% --------------------------------------------------------------

\section{Conclusion}




% --------------------------------------------------------------
%                            Abstract
% --------------------------------------------------------------
\newpage
\thispagestyle{empty}

\vspace*{\fill}
\noindent\rule[2pt]{\textwidth}{0.5pt}\\
{\textbf{Résumé :}}
\lipsum[1]

{\textbf{Mots clés :}}
Lorem ipsum dolor sit amet, consectetur adipiscing elit. Sed non risus. Suspendisse lectus tortor.
\\
\noindent\rule[2pt]{\textwidth}{0.5pt}
\begin{center}
  École centrale de Lyon\\
  36, Avenue Guy de Collongue\\
  code du labo concerné\\
  69134 Écully
\end{center}
\vspace*{\fill}

\end{document}
