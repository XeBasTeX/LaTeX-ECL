% !TEX encoding = UTF-8 Unicode
% -*- coding: UTF-8; -*-
% vim: set fenc=utf-8
\documentclass[a4paper,12pt,french]{article}

\usepackage{centrale}

\hypersetup{
  pdftitle={Titre},
  pdfauthor={Auteur},
  pdfsubject={Sujet},
  pdfproducer={Conversion PDF},
  pdfkeywords={Quelques mots-clés} %
}

\DeclareGraphicsRule{.ai}{pdf}{.ai}{} % Pour insérer des documents .ai
\graphicspath{ {./img/} {./eps/} {./fig/}} % Pour ne pas avoir à ajouter eps/ton-image.jpg

% ------------- Packages spéciaux, nécessaires pour ce rapport, à insérer ici ------------- 
%\usepackage{minted}

\begin{document}

% --------------------------------------------------------------
%                       Page de garde
% --------------------------------------------------------------

\begin{titlepage}
  \begin{center}

    \includegraphics[width=1\textwidth]{logo_ECL_rectangle_quadri_vectoriel.ai}\\[1cm]

    {\large Intitulé de la filière, domaine ou approfondissement}\\[0.5cm]

    {\large Type de projet}\\[0.5cm]

    % Titre
    \rule{\linewidth}{0.5mm} \\[0.4cm]
    { \huge \bfseries Titre du rapport éventuellement en plusieurs lignes \\[0.4cm] }
    \rule{\linewidth}{0.5mm} \\[1.5cm]

    % Auteur(es) et encadrant(es)
    \noindent
    \begin{minipage}{0.4\textwidth}
      \begin{flushleft} \large
        \emph{Auteurs :}\\
        M\up{me} Prénom \textsc{Nom}\\
        M. Prénom \textsc{Nom}\\
        M. Prénom \textsc{Nom}\\
        M. Prénom \textsc{Nom}
      \end{flushleft}
    \end{minipage}%
    \begin{minipage}{0.4\textwidth}
      \begin{flushright} \large
        \emph{Encadrants :} \\
        M.~Prénom \textsc{Nom}\\
        M.~Prénom \textsc{Nom}
      \end{flushright}
    \end{minipage}

    \vfill

    % Bottom of the page
    {\large Version du\\ \today}

  \end{center}
\end{titlepage}

% --------------------------------------------------------------
%                    Table des matières 
% --------------------------------------------------------------

\thispagestyle{empty}
\tableofcontents
%\newpage

% --------------------------------------------------------------
%                         Début du corps
% --------------------------------------------------------------

\section{Introduction}





% --------------------------------------------------------------
%                         Partie 1
% --------------------------------------------------------------

\newpage
\section{Une partie}
\subsection{Une sous-partie}


Pour faire des dérivations :

\[
    \derp{x}{t} + \derd{y}{t} + \derda{z}{t} + \derD{z}{t} = 0
\]



Pour faire des intégrales, le d de la différentielle est écrit avec la commande \texttt{ud} :

\[
    \int_0^t x \ud t = 0
\]



Pour utiliser des opérateurs différentiels:

\[
    \div{x} + \rot{y} + \grad{z} = 0
\]




Pour introduire une figure : 

\begin{figure}[ht!]
    \centering
    \includegraphics[width=0.3\textwidth]{example-image-a}
    \caption{Insérer ici le sous-titre.}
    \label{fig:id-de-la-figure}
\end{figure}




Pour introduire une table : 

\begin{table}[!ht]
\begin{center}
\begin{tabular}{@{}llllllll@{}}
\toprule
Un truc &  &  &  &  &  &  &  \\ \midrule
Un autre &  &  &  &  &  &  & \\ \bottomrule
\end{tabular}
\caption{Un tableau \label{tab:id-de-la-table}}
\end{center}
\end{table}



Et également :

% Pour introduire du code informatique (décommenter)
%\begin{listing}[ht]
%\inputminted[linenos=true, breaklines,frame=lines,framesep=2mm]{python}{code/}
%\caption{Du code informatique}
%\label{listing:id-du-code}
%\end{listing}


% --------------------------------------------------------------
%                            Partie 2
% --------------------------------------------------------------

\section{Une autre partie}
\subsection{Une autre sous-partie}
\lipsum[2]



% --------------------------------------------------------------
%                            Conclusion
% --------------------------------------------------------------

\section{Conclusion}
\lipsum[1]


% --- Biblio par .bib
%\bibliography{ton-fichier-biblio}
%\bibliographystyle{plain-fr.bst}
%\selectlanguage{french}

%\begin{thebibliography}{7}
%\bibitem[Bastien 2019]{id-de-la-source}
%Auteurs : \emph{Un titre},  une date.
%\end{thebibliography}


% --------------------------------------------------------------
%                            Abstract
% --------------------------------------------------------------
\newpage
\thispagestyle{empty}

\vspace*{\fill}
\noindent\rule[2pt]{\textwidth}{0.5pt}\\
{\textbf{Résumé :}}
\lipsum[1]

{\noindent\textbf{Mots clés :}}
Lorem ipsum dolor sit amet, consectetur adipiscing elit. Sed non risus. Suspendisse lectus tortor.
\\
\noindent\rule[2pt]{\textwidth}{0.5pt}
\begin{center}
  École centrale de Lyon\\
  36, Avenue Guy de Collongue\\
  code du labo concerné\\
  69134 Écully
\end{center}
\vspace*{\fill}

\end{document}
